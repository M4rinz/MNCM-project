\ProvidesPackage{AndreaStyle}

%%% ----------------------------------- COMMON PACKAGE IMPORTS -----------------------------------

        % ------ TYPING AND LAYOUT IMPROVEMENTS ------
\RequirePackage{microtype}					% Migliora la tipografia, permettendo ad alcuni elementi di sporgere leggermente
%\RequirePackage[italian]{varioref}			% Per usare il comando \vref{label}, che dà dei collegamenti più dettagliati        <----- NON lo carico qui. Potrei voler scrivere documenti in inglese
\RequirePackage{csquotes}                   % Per gestire le citazioni in modo più avanzato

        % ------ MATHEMATICAL PACKAGES ------
\RequirePackage{mathtools}                    % amsmath sotto steroidi (estende amsmath)
\RequirePackage{amssymb}                      % Simboli matematici extra
\RequirePackage{amsthm}                       % Per creare teoremi e simili
\RequirePackage{physics}                     % Comandi utili per la fisica e la matematica
\RequirePackage{relsize}                     % Per usare comandi come \mathbigger{}, \mathsmaller{}, ecc.
\RequirePackage{mathrsfs}					% Per dei caratteri matematici migliori: \mathscr{} e \mathcal{}
\RequirePackage{mathdots}					% Migliora i tre puntini (per le matrici e le somme)
\RequirePackage{cancel}                      % Per barrare termini nelle espressioni matematiche


        % ------ TABLE-RELATED IMPORTS ------
\RequirePackage{booktabs}                   % Tabelle più belle, in stile booktabs
\RequirePackage{tabularx}                   % Per fare tabelle. Carica il pacchetto array, per gli array.
\RequirePackage{multirow}					% per usare il comando multirow


        % ------ FIGURE AND TABLE CAPTIONS ------

\RequirePackage{caption}  % Per personalizzare le didascalie di figure e tabelle
\RequirePackage{subcaption}  % Per didascalie di sottocomponenti nelle figure e tabelle
\captionsetup{tableposition=top, figureposition=bottom, font=small}


\RequirePackage{xparse}                     % For defining commands with flexible arguments

\RequirePackage{graphicx}                   % Per includere immagini
\RequirePackage{xcolor}                     % Per usare colori nel testo e nelle tabelle
%\RequirePackage{hyperref}                  % Per creare link ipertestuali nel documento.   <------ NON lo carico qui. Va caricato in ciascun documento, per ultimo


% Note: Hyperref should be loaded in the main document, not here, to avoid potential issues.
% Note: varioref is not loaded, as it is language-dependant



%%% --------------------------------------- CUSTOM COMMANDS ---------------------------------------


% Define the \P command with an optional subscript and negative space
\RenewDocumentCommand{\P}{o g g}{
  \IfNoValueTF{#1}{
    % No subscript
    \IfNoValueTF{#2}{
      % No arguments
      \ensuremath{\mathbb{P}}
    }{
      % One argument
      \IfNoValueTF{#3}{
        \ensuremath{\mathbb{P}\left(#2\right)}
      }{
        % Two arguments
        \ensuremath{\mathbb{P}\left(#2 \mid #3\right)}
      }
    }
  }{
    % With subscript and negative space
    \IfNoValueTF{#2}{
      % No arguments
      \ensuremath{\mathbb{P}_{#1}\!}
    }{
      % One argument
      \IfNoValueTF{#3}{
        \ensuremath{\mathbb{P}_{#1}\!\left(#2\right)}
      }{
        % Two arguments
        \ensuremath{\mathbb{P}_{#1}\!\left(#2 \mid #3\right)}
      }
    }
  }
}


% Define the \R command with optional subscript and superscript arguments
\NewDocumentCommand{\R}{o g g}{
  \IfNoValueTF{#1}{
    % No subscript
    \IfNoValueTF{#2}{
      % No superscript
      \mathbb{R}
    }{
      % One superscript argument
      \IfNoValueTF{#3}{
        \mathbb{R}^{#2}
      }{
        % Two superscript arguments
        \mathbb{R}^{#2 \times #3}
      }
    }
  }{
    % With subscript
    \IfNoValueTF{#2}{
      % No superscript
      \mathbb{R}_{#1}
    }{
      % One superscript argument
      \IfNoValueTF{#3}{
        \mathbb{R}_{#1}^{#2}
      }{
        % Two superscript arguments
        \mathbb{R}_{#1}^{#2 \times #3}
      }
    }
  }
}


% Comandi per Argmax e Argmin
\DeclareMathOperator*{\argmax}{arg\,max}    		% per scrivere argmax
\DeclareMathOperator*{\argmin}{arg\,min}    		% per scrivere argmax


% Comando per scrivere la restrizione di una funzione
\newcommand\restr[2]{{											% Command for restriction. We make the whole thing an ordinary symbol
		\left.\kern-\nulldelimiterspace % automatically resize the bar with \right
		#1 % the function
		\vphantom{\big|} % pretend it's a little taller at normal size
		\right|_{#2} % this is the delimiter
}}


% Comando per il complementare di un insieme
\newcommand{\comp}[1]{{#1}^{\complement}}		


% Comando per la differenza simmetrica tra due insiemi
\newcommand{\diffsim}{\mathbin{\scriptstyle\bigtriangleup}}		


% Comando per scrivere le ellipsis [...]
\newcommand*\elide{\textup{[\,\dots]}}



\endinput
