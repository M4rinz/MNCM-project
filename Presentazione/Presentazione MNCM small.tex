\documentclass[10pt,xcolor={table,dvipsnames}]{beamer} 		% carica automaticamente amsthm, amssymb, amsmath, graphicx
\setbeamertemplate{theorems}[numbered]%[ams style] 


\usepackage[T1]{fontenc}				% codifica dei font
\usepackage[utf8]{inputenc}				% lettere accentate da tastiera
\usepackage[italian]{babel}				% lingua del documento
\usepackage[italian]{varioref}			% Per usare il comando \vref{label}, che dà dei collegamenti più dettagliati

% Load the custom style file
\usepackage{AndreaStyle}
% The file `AndreaStyle.sty` is stored in: `D:\Programmi e Applicazioni\texlive\texmf-local\tex\latex\local` for Windows.
% The file `AndreaStyle.sty` is stored in: `/usr/local/texlive/texmf-local/tex/latex/local` for Ubuntu (desktop).
% This won't work in Overleaf, until the AndreaStyle.sty file is added to the project

\usepackage{mathdots}

%\usepackage{algorithm}
%\usepackage[beginLComment=//~,endLComment=~]{algpseudocodex}			% Package for typesetting algorithms

\usepackage{mathrsfs}					% Per dei caratteri matematici migliori: \mathscr{} e \mathcal{}
%\usepackage{braket} 					% Per il comando \Set, e altre (poche) cose
%\usepackage{textcomp}					% Dovrebbe aggiungere più simboli
%\usepackage{bbm}						% Più simboli in \mathbb

\usepackage{fontawesome5}				% Aggiunge simboli da FontAwesome

\usepackage{hyperref}					% Importante: hyperref va caricato nel documento.


%\setcounter{tocdepth}{1}	% profondità dell'indice

	    % TEOREMI CUSTOM:
\theoremstyle{plain}					% Definisce ambienti per Teoremi, esercizi, corollari... Con lo stile adeguato
	\newtheorem{proposizione}{Proposizione}%[section]
	\newtheorem*{proposizione*}{Proposizione}
	
	\newtheorem{teorema}{Teorema}%[section]
	\newtheorem*{teorema*}{Teorema}
		
	%\newtheorem{lemma_es}{Lemma}[esercizio]
	%\newtheorem{lemma}{Lemma}[section]
	\newtheorem*{lemma*}{Lemma}
	\newtheorem{corollario}{Corollario}[section]


\theoremstyle{definition}				
	\newtheorem{definizione}{Definizione}%[section]%[chapter]
	\newtheorem*{definizione*}{Definizione}	%definizione non numerata
	\newtheorem*{notazione}{Notazione}

\theoremstyle{remark}
	\newtheorem{oss}{Osservazione}%[section]
	\newtheorem*{oss*}{Osservazione}


    	% COMANDI CUSTOM
% Define the \indicator command
\NewDocumentCommand{\indicator}{O{t} O{m} O{i}}{%
\mathlarger{\mathbbm{1}}\qty{\scriptstyle {x}_{#1}^{#2}=#3}%
}
% Define the \transpose command
\newcommand{\transpose}[1]{\prescript{t}{}{#1}}
% Define the \Var command, for the variance
\newcommand{\Var}[1]{\operatorname{Var}\qty(#1)}
% Define the \Cov command, for the covariance
\newcommand{\Cov}[1]{\operatorname{Cov}\qty(#1)}

% Define a command to create unnumbered footnotes
\let\svthefootnote\thefootnote
\textheight 1in
\newcommand\blankfootnote[1]{%
\let\thefootnote\relax\footnotetext{#1}%
\let\thefootnote\svthefootnote%
}
% Define the \independent symbol, for independence
\newcommand\independent{\protect\mathpalette{\protect\independenT}{\perp}}
  \def\independenT#1#2{\mathrel{\rlap{$#1#2$}\mkern2mu{#1#2}}}


\usetheme{Madrid}

\title[Seminario MNCM]{Slide di appoggio}			%WIP
%\subtitle{Presentazione e dimostrazione della convergenza} 
\author{Andrea Marino}
\institute[DI UniPi]{Università di Pisa}
%\titlegraphic{\includegraphics[width=2cm]{Immagini/cherubino_black.eps}}
\date[\today]{Metodi Numerici per le Catene di Markov\newline Seminario di fine corso}


% Custom command to insert the summary frame
\newcommand{\insertSummaryFrame}{
    \begin{frame}
        \frametitle{Sommario}
        \tableofcontents[currentsection, subsectionstyle=show/shaded/hide]
    \end{frame}
}

% Show summary at the beginning of each section
\AtBeginSection[]
{
    \begin{frame}
        \frametitle{Sommario}
        \tableofcontents[currentsection, subsectionstyle=show/hide/hide]
    \end{frame}
}

% Show summary at the beginning of each subsection
\AtBeginSubsection[]
{
    \begin{frame}
        \frametitle{Sommario}
        \tableofcontents[currentsection, subsectionstyle=show/shaded/hide]
    \end{frame}
}

\begin{document}
    \begin{frame}[plain]
        \titlepage
    \end{frame}

\section*{Sommario}
	\setcounter{tocdepth}{1}
	\begin{frame}
		\frametitle{Sommario}
		\tableofcontents
	\end{frame}
	
	\setcounter{tocdepth}{2}  
    
\section{Analisi teorica}

    \begin{frame}
        {\hypertarget{frame:teorema_1_part1}{$\hat{P}_{\textup{MoM}}$ è consistente$\qquad 1/2$}}

        Supponiamo che:
        \begin{enumerate}
            \item $\pi_0=\pi$, ossia $\qty{x_t^{(m)}}_{t\in[T]}$ è \emph{fortemente} stazionario
            \item $\P{\vb*{y}}{\vb*{n}}$ rispetti le ipotesi della proposizione~\ref{prop:noise_model}
            \item $N\in\mathbb{N}, A\in\R{S}{S}$ siano noti {\smaller (e costanti: forte stazionarietà $\implies A_t=A\quad\forall\,t\in[T]$)}
        \end{enumerate}

        \begin{teorema}\label{teor:consistenza_P_mom}
            Sia $\hat{P}_{T,K}$ lo stimatore restituito dall'algoritmo per il caso stazionario.

            Nelle ipotesi precedenti, $\hat{P}_{T,K}$ è consistente%, ossia converge in probabilità 
            %a $P$ per $T\to\infty$ e/o $K\to\infty$.
        \end{teorema}
        \begin{block}{Dimostrazione.}
            \begin{itemize}
                \item ${K\to\infty}$
                \begin{itemize}
                    \item Per la legge dei grandi numeri, $\lim_K\hat{\vb*{m}}_t=\mathbb{E}\qty[\vb*{y}_t]$ q.c.
                    Dunque 
                    $\lim_K\hat{\vb*{m}}=\lim_k\frac{1}{T}\sum_{t=1}^T\hat{\vb*{m}}_t=\mathbb{E}\qty[\vb*{y}_1]$ q.c.
                    (cfr. Appendice~\hyperlink{frame:teorema1_lim_mhat:appencide}{\faHandPointRight})
                    \item Similmente si dimostra che $\lim_K\hat{\Sigma}_{t,t+1}=\Cov{\vb*{y}_1,\vb*{y}_2}$ q.c.
                    e dunque $\hat{\Sigma}=\frac{1}{T-1}\sum_{t=1}^{T-1}\hat{\Sigma}_{t,t+1}\overset{K\to\infty}{\longrightarrow}\Cov{\vb*{y}_1,\vb*{y}_2}$ q.c.
                    (cfr. Appendice~\hyperlink{frame:teorema1_lim_Sigmahat:appendice}{\faHandPointRight})
                \end{itemize}
                Ma allora $\hat{P}_{T,K}\overset{K\to\infty}{\longrightarrow}P$ in probabilità.
            \end{itemize}
        \end{block}
    \end{frame}

    \begin{frame}
        {\hypertarget{frame:teorema_1_part2}{$\hat{P}_{\textup{MoM}}$ è consistente$\qquad 2/2$}}

        \begin{block}{}
            \begin{itemize}
                \item $T\to\infty$
                
                Possiamo supporre $K=1$. Dunque $\hat{\vb*{m}}_t=\vb*{y}_t$ e $\hat{\vb*{m}}=\frac{1}{T}\sum_{t=1}^{T}\vb*{y}_t$
            \end{itemize}
        \end{block}

    \end{frame}





% ---------------------------------------------------------------------------------------
% -------------------------------------- APPENDICE --------------------------------------
% ---------------------------------------------------------------------------------------

\section*{Appendice}
	\begin{frame}
		\begin{center}
			\Huge{\textbf{Appendice}}
		\end{center}
	\end{frame}


    \begin{frame}
        {Consistenza di \texorpdfstring{$\hat{P}_{T,K}$}{P_T,K} per \texorpdfstring{$K\to\infty$}{K tendente a infinito}$\qquad 1/2$}
        {\hypertarget{frame:teorema1_lim_mhat:appencide}{Dettagli sul calcolo di \texorpdfstring{$\lim_K\hat{\vb*{m}}$}{un limite}}}

        \begin{block}{}
            Nel calcolo di $\lim_K\hat{\vb*{m}}$ si fa uso dell'ipotesi di stazionarietà:
            \begin{align*}
                \lim_K\hat{\vb*{m}}&=\lim_K\frac{1}{T}\sum_{t=1}^T\hat{\vb*{m}}_t=\frac{1}{T}\sum_{t=1}\
                \lim_K\hat{\vb*{m}}_t=\frac{1}{T}\sum_{t=1}^T\mathbb{E}\qty[\vb*{y}_t]\
                =\frac{1}{T}\sum_{t=1}^T A_t\mathbb{E}\qty[\vb*{n}_t]\\
            \intertext{Per ipotesi $A_t=A\quad\forall\,t\in[T]$. Dunque}
                &=\frac{1}{T}\cdot A\cdot\sum_{t=1}^T\mathbb{E}\qty[\vb*{n}_t]=\frac{1}{T}\cdot A\cdot T\cdot\mathbb{E}\qty[\vb*{n}_1]=A\cdot\mathbb{E}\qty[\vb*{n}_1]=\mathbb{E}\qty[\vb*{y}_1]
            %\intertext{Ma per stazionarietà $\mathbb{E}\qty[\vb*{n}_t]=\mathbb{E}\qty[\vb*{n}_1]\quad\forall\,t\in[T]$. Dunque}
            \end{align*}
            poiché per stazionarietà $\mathbb{E}\qty[\vb*{n}_t]=\mathbb{E}\qty[\vb*{n}_1]\quad\forall\,t\in[T]$.
            
        \end{block}

        \blankfootnote{\textbf{Indietro:}~\hyperlink{frame:teorema_1_part1}{\faHandPointLeft}}
    \end{frame}

    \begin{frame}
        {Consistenza di \texorpdfstring{$\hat{P}_{T,K}$}{P_T,K} per \texorpdfstring{$K\to\infty$}{K tendente a infinito}$\qquad 2/2$}
        {\hypertarget{frame:teorema1_lim_Sigmahat:appendice}{Dettagli sul calcolo di \texorpdfstring{$\lim_k\hat{\Sigma}_{t,t+1}$}{un altro limite}}}

        \begin{block}{}
            Consideriamo 
            \[
                \hat{\Sigma}_{t,t+1}=\frac{1}{K}\sum_{k=1}^{K}\qty(\vb*{y}_t^{(k)}-\hat{\vb*{m}}_t)\cdot\transpose{\qty(\vb*{y}_{t+1}^{(k)}-\hat{\vb*{m}}_{t+1})}=\frac{1}{K}\sum_{k=1}^{K}\qty(\vb*{y}_t^{(k)}\cdot\transpose{\vb*{y}}_{t+1}^{(k)})-\hat{\vb*{m}}_t\cdot\transpose{\hat{\vb*{m}}_{t+1}}
            \]
            Per la legge dei grandi numeri \textcolor{red}{\textbf{FINIRE}}
            
        \end{block}

        \blankfootnote{\textbf{Indietro:}~\hyperlink{frame:teorema_1_part1}{\faHandPointLeft}}
    \end{frame}



\end{document}